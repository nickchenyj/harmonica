\documentclass[10pt,twocolumn]{article}
\title{The Harmonica Processor Core}
\author{Chad Kersey}

%\usepackage[margin=1in]{geometry}
\usepackage{graphicx}

\begin{document}
\maketitle

\begin{figure*}
\begin{center}
\includegraphics[width=6in]{fig/arch}
\caption{Block diagram of the Harmonica microarchitecture. The set of functiona\
l units is arbitrary.}
\label{fig:arch}
\end{center}
\end{figure*}


\section{Microarchitecture}
Harmonica is an in-order, pipelined implementation of a significant subset of the Harp instruction set architectures supporting out-of-order instruction completion.
It is a parameterized implementation with variable width, variable register file dimensions, and interchangeable functional units.

The Harmonica pipeline can be divided into four phases:
\begin{itemize}
  \item Instruction Fetch
  \item Instruction Decode/Issue
  \item Execution
  \item Writeback
\end{itemize}

Fetch, Issue, and Writeback each require a single cycle.
Execution time can vary depending on the functional unit design.

\subsection{Instruction Identifiers (IIDs)}
A small register, large enough to contain a number uniquely identifying each fetched instruction still within the pipeline, is contained in the fetch stage.
This number, the Instruction Identifier (IID), uniquely describes each instruction in the pipeline.
They are used in the writeback stage to ensure that only the most recently-issued instruction writing to a register will influence its value.
This enables out-of-order completion of instructions as long as issue remains in-order.

Fetch, Issue, and Writeback each require a single cycle.
Execution time can vary depending on the functional unit design.

\subsection{Parameterizability}
Architecture parameters, such as the number of registers and the width of the datapath, are all software configurable.
At the top of \texttt{harmonica.cpp} a set of constants is defined, describing the instruction set parameters.
Between these constants and the configurable set of functional units, Harmonica can be reconfigured for specific applications.

\section{Functional Units}

The functional units are implemented in \texttt{funcunit.h}.
Inputs to the functional units are of type \texttt{fuInput} and outputs are of type \texttt{fuOutput}.
The set of opcodes provided by a functional unit are provided (as a vector of \texttt{unsigned int}s by a member function called \texttt{get\_opcodes()}.

\subsection{Arithmetic Logic Unit}

\texttt{BasicAlu} is a simple ALU implementation, supporting all common integer arithmetic and logic functions involving registers and immediates which can be handled in a single cycle.
The sole exceptions are multiplication and division.

\subsection{Predicate Logic Unit}

\texttt{PredLu} performs all predicate operations, including predicate logic and conversions from register vales to predicate values.

\subsection{Load/Store Unit}

Only a simple LSU, \texttt{SramLsu}, has been implemented so far.
This is a very simple unit with a single cycle latency, storing all data in a (configurable size, but usually tiny) synchronous SRAM.

\section{Performance}
The prototype scheduling logic has a critical path length of 68 NAND gates.
Without technology mapping or placement/routing, it is difficult to say how this translates to delays in terms of FO4, but assuming each NAND gate has a delay on the order of ten picoseconds leads to a clock frequency on the order of one gigahertz.

\end{document}
